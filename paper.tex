% Paper structure:
%
% Introduction
%  - Presentation of the ontology.
%  - Example of a guideline.
%  - What is an annotated table?
%  - Motivating example.
%
% Proposal
%  - Formalization of a guideline using SPARQL queries
%  - Extending the ontology to represent constraints
%  - Implementation
%  - Experimentation
%  - Related work
%  - Conclusions
%  - Future work
%    - Represent positive constraints for classification purposes

\documentclass[a4paper, 10pt]{article}
\usepackage[margin=1.5in]{geometry}
\usepackage[utf8]{inputenc}
\usepackage{amssymb}
\usepackage{hyperref} % Needed for \href
\usepackage{xspace} % Needed for \xspace
\usepackage{graphicx} % Needed for \includegraphics
\usepackage{listings} % Needed for the lstlisting environment
\usepackage{caption} % Needed for \captionsetup
\usepackage{float} % Needed for figures with the H placement specifier
\usepackage[textsize=tiny, textwidth=8em]{todonotes} % Needed for \todo
\usepackage{parskip} % Prevents itemize blocks from being affected when
                     % overriding \parskip

% Utility macros
\newcommand{\atweb}{\textbf{@Web}\xspace}
\newcommand{\nary}{$n$-ary\xspace}
\newcommand{\fnhref}[2]{\href{#2}{#1}\footnote{\url{#2}}}
\newcommand{\ifnhref}[2]{\textit{\fnhref{#1}{#2}}}
\newcommand{\code}[1]{\texttt{#1}}
\newcommand{\img}[3]{
  \begin{figure}[H]
    \centering
    \includegraphics[width=\textwidth]{img/#1}
    \caption{#2}
    \label{#3}
  \end{figure}
}

% Paragraph formatting
\setlength{\parskip}{1em}

% Prevent hyphenation
\tolerance=1
\emergencystretch=\maxdimen
\hyphenpenalty=10000
\hbadness=10000

% Bullets in nested \itemize environments
\renewcommand{\labelitemii}{$\circ$}
\renewcommand{\labelitemiii}{---}

% lstlisting environment appearance
\lstset{
  basicstyle=\ttfamily\small,
  escapeinside={(*@}{@*)},
  numbers=left
}
\captionsetup[lstlisting]{position=bottom}

%%%%%%%%%%%%%%%%%%%%%%%%%%%%%%%%%%%%%%%%%%%%%%%%%%%%%%%%%%%%%%%%%%%%%%%%%%%%%%%%

\begin{document}

\begin{abstract}
  In this paper we study the use of SPARQL queries to verify a set of integrity
  constraints in an ontology dedicated to the annotation of experimental data
  in biorefinery, packaging and other domains.

  \vspace{1em}

  \textbf{Keywords}: Semantic Web, RDF, OWL, SPARQL, Integrity constraints.
\end{abstract}

\section{Introduction}
\label{sec:introduction}

The \ifnhref{@Web platform}{https://www6.inra.fr/cati-icat-atweb} is a Semantic
Web application that allows domain experts to annotate experimental data in
scientific documents, and researchers to explore and query those annotations
via a graphical user interface. The annotations are stored in a publicly
accessible \ifnhref{RDF}{http://www.w3.org/TR/rdf11-primer/} graph using a
vocabulary predefined in an \ifnhref{OWL}{http://www.w3.org/TR/owl-primer/}
ontology, and shared with the research community.

Given the error-prone nature of the data annotation process, a set of integrity
constraints has been identified that all annotated data must fulfill. It is
desired to validate these constraints automatically and report any validation
errors to the domain expert during the data annotation process.

\todo{
  Do we really need to mention the comparison we did between SPARQL, SHACL and
  Shape Expressions? \\
  \vspace{1em}
  Shouldn't we focus on SPARQL instead and leave out the details on why we
  chose SPARQL over the other alternatives?
}
To this end, we first surveyed the current W3C recommendations for constraining
the contents of RDF graphs, and the available tools implementing these
recommendations. We decided to focus our analysis on
\ifnhref{SPARQL}{http://www.w3.org/TR/sparql11-query/}, \ifnhref{Shape
Expressions}{http://www.w3.org/2013/ShEx/Primer} and
\ifnhref{SHACL}{http://www.w3.org/TR/shacl/}. We then implemented a set of test
constraints using each of the available tools and compared them according to
expressiveness, verbosity, readability, etc.

Our analysis shows that none of the libraries implementing the Shape
Expressions recommendation fully support all our use cases. We also observe
that certain constraints expressed in SHACL require nesting SPARQL queries that
are comparable in length to stand-alone SPARQL queries implementing those same
constraints, thus defeating the purpose of an alternate constraint language.

We finish our analysis by comparing the running times of a set of test
constraints implemented as SPARQL queries against different triple stores
supported by \ifnhref{Jena}{https://jena.apache.org/}, a Java library for
building Semantic Web applications which is already used in @Web for other
purposes.


\subsection{The \atweb platform}

The \atweb platform is a software system that allows researchers to extract
heterogeneous experimental data from tables in scientific publications (such as
papers, articles, CSV files, etc.) and store it in an RDF graph following a
uniform structure. The platform provides access to the RDF graph and also
graphical tools to explore and query the data.

Some typical examples of data extracted from scientific publications using the
\atweb platform include input and output parameters associated to a milling
operation in a multi-step biorefinery experiment, the kind of biomass
associated to that experiment, chemical properties such as glucose rate, oxygen
permeability, etc.


\subsection{Ontology}
\label{sec:ontology}

Internally, \atweb stores data in an RDF graph following a vocabulary defined
in an OWL ontology. In this ontology, a model for \nary relations in
quantitative experimental data is established. Under this model, \nary
relations have one or more input parameters controlling various aspects of the
experiment, and exactly one output parameter.

Conceptually, the ontology can actually be thought of as two separate
ontologies:

\begin{itemize}
  \item a core ontology, where a \textit{Relation} OWL class is defined, along
    with object properties \textit{hasInput} and \textit{hasOutput},
    cardinality constraints (each relation has one or more input parameters and
    exactly one output parameter), etc.

  \item a domain ontology with \textit{Relation} subclasses for each kind of
    experimental data relevant for a particular domain. Some examples in the
    biorefinery domain include milling processes, extrusion processes,
    enzymatic treatments, etc.
\end{itemize}

A number of additional concepts are defined in both ontologies such as
magnitudes and units of measurement, but for the sake of brevity we won't
elaborate any further.

\autoref{fig:relation} shows an example of an \nary relation representing a
milling step in an experiment in the biorefinery domain.


\img{relation.jpg}
    {
      \textit{Milling solid quantity output} relation: an example of an \nary
      relation in \atweb.
    }
    {fig:relation}

Finally, the \atweb platform uses concepts defined in an auxiliary OWL ontology
to store metadata associated to scientific publications such as document title,
authors, table title and number, etc.


\subsection{Annotated tables}

The screenshot in \autoref{fig:annotated-table} illustrates a typical annotated
table in the \atweb platform after it's been extracted from a scientific
publication, which in this case belongs to the biorefinery domain.

\img{annotated-table.jpg}
    {An example of an annotated table in \atweb.}
    {fig:annotated-table}

The table contains data from 4 different experiments (see column
\textit{Experience number}), each composed of many steps (see column
\textit{Process step number}). Each row in this table represents a step in an
experiment, and for each row there is an instance of an \nary relation in the
underlying RDF graph. Rows 1 and 3, for example, correspond to cutting milling
and wet disk milling steps, respectively, and both are associated with
instances of the \textit{milling solid quantity output} relation.


\subsection{Guidelines}

In the \atweb platform, each relation is associated with a set of guidelines
written in natural language. These guidelines explain the kinds of experiments
a relation is meant to represent and in many cases provide a number of rules
that all instances of a relation are supposed to fulfill.

\autoref{fig:guideline} shows capture of the screen in \atweb where an
annotator can read the guidelines associated to the \textit{milling solid
quantity output} relation, introduced in \autoref{sec:ontology}.

\img{guideline.jpg}
    {
      Guidelines associated to the \textit{milling solid quantity output}
      relation.
    }
    {fig:guideline}


\subsection{Integrity constraints}
\label{sec:integrity-constraints}

The guideline highlighted in yellow in \autoref{fig:guideline} represents
a rule that must be valid for all instances of the \textit{milling solid output
quantity} relation. We call this kind of rule an \textit{integrity constraint}.

In most cases, integrity constraints can be stated formally as mathematical
equations or pseudocode. To show this, the guideline in \autoref{fig:guideline}
highlighted in yellow is transcribed below, followed by a possible
formalization as a mathematical equation.

Guideline:

\begin{center}
  \textit{``The output quantity of a step is equal to the sum of the quantity
  of water used and the quantity of biomass present in the step.''}
\end{center}

Equation:

\begin{center}
  $output = waterInput + biomassInput$
\end{center}

Once a guideline is stated formally, it should be possible to verify
automatically whether it is being fulfilled by an instance of its associated
relation.

\subsection{Problem statement}

The goal of this work is, therefore, to identify a technology or technique that
allows expressing integrity constraints formally, and automatically verifying
whether an \nary relation instance satisfies an integrity constraint.


\section{Proposed solution}

We propose using SPARQL queries for expressing and verifying integrity
constraints. Specifically, given an integrity constraint $c$ and its associated
relation $r$, we propose to express $c$ as a SPARQL query that will select all
instances of $r$ that do not fulfill $c$.

Our decision to focus on SPARQL was based on the following facts:

\begin{itemize}
  \item \todo{Refine this list.} most general tool
  \item mature specification
  \item well supported, efficient implementations available
  \item strong community around it
\end{itemize}

In the following sections we explain in detail our implementation of the
proposed solution, we give examples of actual integrity constraints expressed
this way, and we show what the implementation looks like from the \atweb user's
point of view.


\subsection{Integrity constraints expressed as SPARQL queries}
\label{sec:integrity-constraints-as-sparql}

In order to make sense of the result set returned by the execution of a SPARQL
query implementing an integrity constraint, some assumptions have to be made.

\todo{Should we mark this as a definition?} Let $c$ be an integrity constraint
associated to a relation $r$, and let $q$ be a SPARQL query implementing $c$.
We make the following assumptions about $q$:

\begin{itemize}
  \item Each solution in a result set obtained by evaluating $q$ corresponds to
    exactly one instance of $r$, which we will call $i$.

  \item Each solution in the result set must include a variable
    \code{relation} which contains the IRI of $i$. All other variables will
    be ignored.

  \item $i$ must be in violation of $c$.

  \item There's at least one solution in the result set corresponding to each
    instance of $r$ in violation of $c$.
\end{itemize}

\todo{Decide if it's worth talking about the structure of queries instead of
showing a concrete example.}
SPARQL queries implementing integrity constraints will therefore have a
structure similar to the example in \autoref{lst:structure-example}.

\begin{minipage}{\textwidth}
\begin{lstlisting}[label={lst:structure-example},
                   caption={Structure of a SPARQL query implementing an
                            integrity constraint.}]
SELECT ?relation
WHERE {
  ?relation a example:MyRelationClass .
  # ... restrictions go here
}
\end{lstlisting}
\end{minipage}

\todo{Add a concrete toy example. Idea: input parameter must be greater than
output parameter.} Let's illustrate this with an example. Let $r$ be a relation
representing ...


\subsection{An example of a real integrity constraint implemented as a SPARQL
query}

In this section we will show how the integrity constraint mentioned in
\autoref{sec:integrity-constraints} is implemented in the \atweb platform.

Recall from \autoref{sec:integrity-constraints} the integrity constraint in
question and its mathematical formulation:

\begin{center}
  \textit{``The output quantity of a step is equal to the sum of the quantity
  of water used and the quantity of biomass present in the step.''}

  $output = waterInput + biomassInput$
\end{center}

In order to express this integrity constraint as a SPARQL query we must, first,
select the nodes in our RDF graph corresponding to the $output$, $waterInput$
and $biomassInput$ variables in the equation above. \autoref{fig:rdf-graph}
shows what a subset of our RDF graph containing an instance of the
\textit{milling solid quantity output} relation would look like. We're
therefore interested in selecting the nodes painted in a light blue color.

\img{rdf-graph.png}
    {
      Subset of an RDF graph containing an instance of the \textit{milling
      solid quantity output} relation.
    }
    {fig:rdf-graph}

\todo{Should we explain nodes and edges relative to the annotation ontology?
(hasForTable, hasForRow, etc.) Should we explain fuzzy sets?} We're now ready
to introduce the SPARQL query implementing this integrity constraint, which you
can see in \autoref{lst:sparql-query}.

\begin{minipage}{\textwidth}
\begin{lstlisting}[label={lst:sparql-query},
                   caption={A SPARQL query implementing an integrity
                            constraint.}]
SELECT ?relation ?solid_qty ?liquid_qty ?output_qty
       ?docid ?doctitle ?tableid ?tabletitle ?rownum
WHERE {
?doc anno:hasForID ?docid ;
     dc:title ?doctitle ;
     anno:hasTable ?table .

?table anno:hasForID ?tableid ;
       dc:title ?tabletitle ;
       anno:hasForRow ?row .

?row anno:hasForRowNumber ?rownum ;
     anno:hasForRelation ?relation .

?relation a bioraf:milling_solid_quantity_output_relation ;
          core:hasAccessConcept ?solid ;
          core:hasAccessConcept ?liquid ;
          core:hasResultConcept ?output] .

?solid a bioraf:biomass_quantity ;
       anno:hasForFS [a anno:Scalar ;
                      anno:hasForFuzzyElement /
                      anno:hasForMaxKernel ?solid_qty] .

?liquid a bioraf:water_quantity ;
        anno:hasForFS [a anno:Scalar ;
                       anno:hasForFuzzyElement /
                       anno:hasForMaxKernel ?liquid_qty] .

?output a bioraf:output_solid_constituent_quantity ;
        anno:hasForFS [a anno:Scalar ;
                       anno:hasForFuzzyElement /
                       anno:hasForMaxKernel ?output_qty] .

FILTER (xsd:float(?output_qty) != xsd:float(?solid_qty) + xsd:float(?liquid_qty))
}
\end{lstlisting}
\end{minipage}

A few things to note about this query:

\begin{itemize}
  \item A variable \code{relation} is selected, which is bound on line 15 to an
    instance of the \textit{milling solid output quantity} relation, as per the
    convention explained in \autoref{sec:integrity-constraints-as-sparql}.
    Other variables are also selected for debugging purposes, such as the title
    of the scientific publication and the table from which this relation
    instance was extracted.

  \item Nodes for water, biomass and output quantities are selected on lines
    23, 28 and 33, respectively. These are the nodes painted in light blue on
    \autoref{fig:rdf-graph}.

  \item Line 35 discards instances of the \textit{milling solid output
    quantity} relation that satisfy the equation $output = waterInput +
    biomassInput$, selecting only those instances that are in violation with
    the integrity constraint.
\end{itemize}


\subsection{Extending the ontology to represent constraints}

Pending.


\subsection{Implementation}

Pending.


\subsection{Experimentation}

Pending.


\subsection{Related work}

Pending.


\section{Conclusions}

Pending.


\section{Future work}

Pending.


% \begin{thebibliography}{9}
%
% \bibitem{lamport94}
%   Leslie Lamport,
%   \emph{\LaTeX: a document preparation system},
%   Addison Wesley, Massachusetts,
%   2nd edition,
%   1994.
%
% \end{thebibliography}

\end{document}
